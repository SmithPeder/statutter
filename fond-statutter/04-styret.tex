\section{Fondstyret}
\subsection{Om Fondstyret}
Fondstyret er Abakus’ fonds øverste organ etter generalforsamlingen og styrer
Abakus’ fond i henhold til Abakus’ fonds statutter.

\subsection{Fondstyrets sammensetning}
Fondstyret består av et æresmedlem, et ordensmedlem, en tidligere økonomiansvarlig fra
Hovedstyret, et tidligere medlem av Hovedstyret, Abakus-medlem 1 og Abakus-medlem 2.
Ingen av Fondstyrets medlemmer kan ha aktive verv i Hovedstyret. Abakus-medlem 1 og 2 skal
være medlemmer som ikke innehar tittelen æresmedlem.

\subsection{Valg av fondstyre}
\subsubsection{}
Fondstyret skal velges på generalforsamling. Søknad fra interesserte kandidater
leveres hovedstyret to -2- uker før ordinær generalforsamling.

\subsubsection{}
Tilstedeværende stemmeberettigede kan benke kandidater til stillinger som skal velges
på generalforsamlingen fra og med generalforsamlingen er satt, til avstemning holdes for
den aktuelle stillingen.

\subsubsection{}
Det er Abakus’ Hovedstyre sin oppgave å passe på at det er tilstrekkelig med kandidater til
stillingene slik at Fondstyrets sammensetning kan godkjennes. Hovedstyret skal arbeide for
at ett av de to Abakus-medlemmene i Fondstyret ikke skal være nåværende eller tidligere
medlem av en komité i Abakus.

\subsubsection{}
Nytt fondstyre inntrer 1. mai.

I oddetallsår skal følgende styremedlemmer velges på nytt: Ordensmedlem, et tidligere
Hovedstyremedlem og Abakus-medlem 1.

I partallsår skal følgende styremedlemmer velges på nytt: Æresmedlem, en tidligere
økonomiansvarlig i Hovedstyret og Abakus-medlem 2.

Valgte kandidater skal godkjennes av generalforsamlingen. Æresmedlemmer, Ordensmedlemmer,
tidligere Hovedstyremedlemmer og tidligere økonomiansvarlig i Hovedstyret kan
stille til gjenvalg til Fondstyret.
