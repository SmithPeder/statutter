\section{Fondstyret}
\subsection{Om Fondstyret}
Fondstyret er Abakus’ fonds øverste organ etter generalforsamlingen og styrer
Abakus’ fond i henhold til Abakus’ fonds statutter.

\subsection{Fondstyrets sammensetning}
Fondstyret består av et ordensmedlem, en tidligere økonomiansvarlig fra
Hovedstyret, et tidligere medlem av Hovedstyret og tre Abakus-medlemmer.
Ingen av Fondstyrets medlemmer kan ha aktive verv i Hovedstyret. To av de tre Abakus-medlemmene
skal ikke være medlemmer som innehar tittelen æresmedlem, ordensmedlem eller tidligere Hovedstyre-medlem.

\subsection{Valg av fondstyre}
\subsubsection{}
Fondstyret skal velges på generalforsamling. Søknad fra interesserte kandidater
leveres hovedstyret to -2- uker før ordinær generalforsamling.

\subsubsection{}
Tilstedeværende stemmeberettigede kan benke kandidater til stillinger som skal velges
på generalforsamlingen fra og med generalforsamlingen er satt, til avstemning holdes for
den aktuelle stillingen.

\subsubsection{}
Det er Abakus’ Hovedstyre sin oppgave å passe på at det er tilstrekkelig med kandidater
til stillingene slik at Fondstyrets sammensetning kan godkjennes.

Dersom ingen kandidat får flertallet av stemmene må Hovedstyret stille en eller
flere ny(e) kandidater innen én -1- uke. Ekstraordinær generalforsamling skal
finne sted senest to -2- uker etter det underkjente valget. Tidsfrister for
ordinært kandidatvalg gjelder ikke i dette tilfellet.

\subsubsection{}
Hovedstyret skal arbeide for at ett av de tre Abakus-medlemmene i Fondstyret ikke skal være nåværende eller tidligere medlem av en komité i Abakus,
og at det stiller kandidater som planlegger å oppholde seg i Trondheim under vervperioden.

\subsubsection{}
Nytt fondstyre trer i kraft fra og med seks -6- uker etter det blir valgt, men datoen kan flyttes opp til to -2- uker hvis Fondet ser dette som hensiktsmessig.

I oddetallsår skal følgende styremedlemmer velges på nytt: Ordensmedlem, et tidligere
Hovedstyremedlem og et Abakus-medlem.

I partallsår skal følgende styremedlemmer velges på nytt: en tidligere
økonomiansvarlig i Hovedstyret og to Abakus-medlemmer.

Det er medlemmer som har sittet i Fondstyret i to -2- år som erstattes ved hvert valg.

Valgte kandidater skal godkjennes av generalforsamlingen. Ordensmedlemmer,
tidligere Hovedstyremedlemmer og tidligere økonomiansvarlig i Hovedstyret kan
stille til gjenvalg til Fondstyret.
