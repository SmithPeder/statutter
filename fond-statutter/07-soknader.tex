\section{Krav til godkjenning av søknader}
\subsection{Om søknadenes størrelse og godkjenning}
En søknad om fondets midler skal sendes til fondets styre. Søknaden skal være
på minst 10 000 kr. Hovedstyret kan videresende mottatte søknader til fondet,
uavhengig av beløpets størrelse. Søknader på mindre enn 250 000 kr og 25 \% av
fondets størrelse, kan behandles av fondstyret. En slik søknad godkjennes av
fondstyret ved kvalifisert flertall. Søknader på
større beløp enn dette skal behandles av fondets generalforsamling. Søknaden
godkjennes av generalforsamlingen ved kvalifisert flertall.

\subsection{Årlige beløpsbegrensninger}
Det er ingen årlige begrensninger på hvor mange søknader fondstyret kan
godkjenne. Totalsummen av godkjente søknader kan hverken overstige 300 000 kr
eller 30 \% av fondets størrelse. Dersom det ønskes å bruke mer enn dette må
det godkjennes av generalforsamlingen.

\subsection{Hvem som kan søke om midler}
Det er kun medlemmer og komitéer av Abakus som har anledning til å søke om
fondets midler. 

\subsection{Søknadens innhold}
Søknaden skal inneholde hvem som søker, formålet med søknaden og antall kroner
det søkes om. Søknaden skal være velbegrunnet og ha som hensikt om å komme
flest mulige medlemmer av Abakus til gode. 
