\section{Organisasjon og verv}
\subsection{Om Abakus}

\subsubsection{}
Abakus består av:

\begin{itemize}
\item Styrene: Hovedstyret og Fondstyret
\item Gruppene: Inter, Sosialt, Media, Diverse
\item Komiteene: Arrkom, backup, Bedkom, Fagkom, Koskom, LaBamba, PR, readme og Webkom
\item Abakusrevyen
\item Undergrupper og interessegrupper
\item Øvrige medlemmer
\end{itemize}

\subsubsection{}
Ingen kan pålegges verv uten samtykke.

\subsubsection{}
Komiteene er gruppert i følgende grupper:

\begin{itemize}
\item Sosialt: Arrkom, Koskom, LaBamba
\item Inter: Fagkom, Bedkom
\item Media: PR, readme, Webkom
\item Diverse: backup
\end{itemize}

\subsection{Hovedstyret}

\subsubsection{}
Hovedstyret er Abakus øverste organ etter generalforsamlingen og skal styre
Abakus i henhold til statuttene.

\subsubsection{}
Foreningens Hovedstyre består av leder, nestleder, økonomiansvarlig og de
gruppeansvarlige. Ingen som innehar stilling i Hovedstyret kan sitte som leder i noen komité.

\subsubsection{}
Nestleder er, med leders samtykke, leders stedfortreder i leders fravær.

\subsubsection{}
Vedtak som fattes på Hovedstyrets møter krever absolutt 50 \% flertall. I
tilfelle stemmelikhet har foreningens leder dobbeltstemme.

\subsubsection{}
Hovedstyret kan etter behov nedsette utvalg for å avlaste Hovedstyret på
særskilte områder eller oppgaver. Retningslinjer og vedtekter for utvalgene
fastsettes av Hovedstyret, og utvalget svarer til Hovedstyret. Hovedstyret
redegjør for utvalgets virksomhet i forbindelse med årsberetningene (jfr. §
8.5.2).

\subsubsection{}
Hovedstyret skal føre fullstendige referater fra alle sine styremøter. Disse
skal fremlegges på foreningens nettside senest én -1- uke etter møtet.
Hovedstyret kan utelate informasjon fra referatene dersom de finner det
hensiktsmessig.

\subsubsection{}
De gruppeansvarlig har ansvaret for forsvarlig drift av sine respektive komiteer (jf. §9.1.3).
Nestleder fungerer som gruppeansvarlig for Diverse-gruppen.

\subsection{Abakus’ fond}

\subsubsection{}
Abakus skal ha et fond adskilt fra Hovedstyret. Fondet heter Abakus’ fond og forvaltes av Fondstyret.

\subsubsection{}
Abakus’ fonds formål er å gi Abakus en økonomisk trygghet og sørge for at Abakus sitt
økonomiske overskudd fordeles rettferdig på nåværende og fremtidige medlemmer av Abakus.

\subsubsection{}
Fondstyret består av et ordensmedlem, en tidligere økonomiansvarlig fra Hovedstyret, et tidligere
medlem av Hovedstyret og tre Abakus-medlemmer. Ingen av Fondstyrets medlemmer kan ha
aktive verv i Hovedstyret. To av de tre Abakus-medlemmene skal ikke være medlemmer som
innehar tittelen æresmedlem, ordensmedlem eller tidligere Hovedstyre-medlem.

\subsubsection{}
Valg til Fondstyret gjennomføres på samme måte som valg til leder, nestleder og
økonomiansvarlig (ref. § 8.4).

\subsubsection{}
Vervet i Fondstyret varer to år. Tre av stillingene er til valg hver ordinære generalforsamling.

\subsubsection{}
En oppløsning av fondet kan kun skje dersom et enstemmig fondstyre og generalforsamlingen
ved kvalifisert flertall stemmer for. Ved oppløsning skal fondets investeringer selges, og stå
uberørt på en høyrentekonto i tre -3- år, dette for å oppfordre til gjenopptak av fondet. Dersom
det går tre -3- år etter oppløsningen uten at fondet blir gjenopptatt, tilfaller fondets midler
Hovedstyret.

\subsection{Kasserere}

\subsubsection{}
Kasserer i den enkelte komité er ansvarlig for den økonomiske styringen til
komiteen, og sørger for at komiteen holder seg innenfor de økonomiske rammene satt av budsjettet.
Vedkommende sitter som ansvarlig for at driften er forsvarlig.

\subsubsection{}
Hvis kasserer i en komité avslutter sitt verv før regnskapsårets slutt, skal Hovedstyret sørge for at regnskapet sluttføres og godkjenne det før påtroppende kasserer overtar.

\subsubsection{}
Alle kasserere må underskrive en bindende kassereravtale med foreningen før tiltredelse i vervet.

\subsubsection{}
Hovedstyret skal velge en vara for økonomiansvarlig i Hovedstyret blant kassererne eller tidligere økonomiansvarlige i Hovedstyret.

\subsection{Undergrupper og interessegrupper}

\subsubsection{}
Undergrupper er økonomisk og administrativt underlagt Hovedstyret. Nestleder i
Hovedstyret har ansvaret for gruppene. Undergruppene administrerer sine egne
opptak.

\subsubsection{}
Interessegrupper er økonomisk og administrativt underlagt backup. Medlemskap i
interessegruppene skal være tilgjengelig for alle Abakus-medlemmer.
