\section{Generalforsamling}
\subsection{Om generalforsamlingen, innkallinger og frister}
\subsubsection{}
Generalforsamlingen er Abakus sitt høyeste organ og skal avholdes minst én
gang i året.

\subsubsection{}
Ordinær generalforsamling skal finne sted i den første uken av mars, men kan
flyttes opp til to -2- uker hvis Hovedstyret ser dette som hensiktsmessig.

\subsubsection{}
Ekstraordinær generalforsamling skal innkalles når Hovedstyret eller minst
ti -10- av foreningens medlemmer krever det.

\subsubsection{}
Enhver generalforsamling bekjentgjøres minst én -1- måned i forveien, med
nyhet på foreningens nettside. Det skal også promoteres for generalforsamlingen.

\subsubsection{}
Hovedstyret skal i forkant av valget lyse ut stillingene leder, nestleder og
økonomiansvarlig innad i Abakus minst én -1- måned før generalforsamlingen.
Søknad fra interesserte kandidater leveres Hovedstyret to -2- uker før
generalforsamlingen. Ny leder, nestleder og økonomiansvarlig skal ha deltatt
aktivt i en av Abakus’ komiteer eller utvalg i minst tre -3- måneder i forkant
av valget, såfremt en slik kandidat stiller til valg.

\subsubsection{}
Forslag til endring av statuttene skal være Hovedstyret i hende minst to -2-
uker før annonsert generalforsamling. Hovedstyret skal informere om alle
mottatte forslag på generalforsamlingen.

\subsubsection{}
Komiteene må ha både sine nye styrekandidater, og reservekandidater, klare minst to -2- uker før ordinær
generalforsamling. Styrekandidaten velges ved et årlig valg internt i hver enkelt komité.

\subsubsection{}
Saksliste og sakspapirer skal være tilgjengelig for foreningens medlemmer minst
én -1- uke før annonsert generalforsamling. Dette innebærer publisering på
foreningens nettside.


\subsection{Beslutningsdyktighet}
\subsubsection{}
En beslutningsdyktig generalforsamling defineres ved et fremmøte på minimum
tretti -30- medlemmer. Minst seks -6- fra Hovedstyret plikter å møte på
generalforsamlingen, samt minst to -2- medlemmer som ikke sitter i Hovedstyret
fra hver komité.

\subsubsection{}
Alle foreningens medlemmer er stemmeberettigede.

\subsubsection{}
Dersom ikke annet er nevnt i statuttene, avgjøres en avstemning i Generalforsamlingen ved
alminnelig flertall.

\subsubsection{}
Vedtak fattet av en generalforsamling kan bare annulleres av en generalforsamling.

\subsubsection{}
Under generalforsamling tillates ikke bruk av fullmakt ved stemmegivning.

\subsection{Statuttendringer}
\subsubsection{}
Statuttendringer krever kvalifisert flertall ved en generalforsamling.

\subsubsection{}
Vedtatte statuttendringer trer i kraft i det generalforsamlingen heves.

\subsubsection{}
Generalforsamlingen har til enhver tid mulighet til å endre på oppsett,
utforming, skrivefeil og formuleringer i statuttforslag, så lenge dette ikke
endrer statuttforslagets opprinnelige intensjon.

\subsubsection{}
Hovedstyret har til enhver tid myndighet til å endre oppsett, utforming og
formuleringer samt rette skrivefeil i Abakus sine statutter såfremt disse ikke
endrer innholdet og betydningen i de aktuelle statuttene. Eventuelle
endringsforslag må offentliggjøres 2. april eller 1. oktober.

Hovedstyret må redegjøre for de endringer de har gjort ved å offentliggjøre
disse på foreningens nettside. Dersom et medlem av Abakus gir skriftlig uttrykk
til Hovedstyret innen to -2- uker om at en endring ikke ivaretar statuttens
opprinnelige intensjon, må endringen behandles på generalforsamling. Endringer
som ikke blir disputert i løpet av denne perioden trer i kraft to -2- uker
etter offentliggjøringen.

\subsubsection{}
Generalforsamlingen kan ikke overstyre statuttene, med mindre det er annonsert på forhånd som beskrevet i §8.3.6.

\subsubsection{}
Dersom et av Abakus' medlemmer ønsker å overstyre en statutt under
generalforsamlingen må dette ønsket være Hovedstyret i hende minst to -2- uker
før generalforsamlingen. Ønsker om overstyring av en statutt under
generalforsamlingen må på forhånd opplyses om i sakspapirene. Hver enkelt
overstyring skal stemmes over under generalforsamlingen og må godkjennes ved
kvalifisert flertall.

\subsection{Valg av leder, nestleder og økonomiansvarlig}
\subsubsection{}
Ny leder, nestleder og økonomiansvarlig velges av generalforsamling ved
alminnelig flertall. Så lenge ingen kandidater oppnår alminnelig flertall
fjernes den kandidaten med lavest antall stemmer og det stemmes på nytt. Blanke
stemmer teller og kan underkjenne kandidatene.

\subsubsection{}
Hvis det til slutt ikke gjenstår noen kandidater kan generalforsamlingen
foreslå nye kandidater til valg. Dersom ingen
kandidat får flertallet av stemmene må Hovedstyret stille en eller flere ny(e)
kandidater innen én -1- uke. Ekstraordinær generalforsamling skal finne sted
senest to -2- uker etter det underkjente valget. Tidsfrister for ordinært
kandidatvalg gjelder ikke i dette tilfellet.

\subsubsection{}
Kandidater som har stilt til valg som leder, nestleder eller økonomiansvarlig
og blitt underkjent av generalforsamlingen, enten ordinær eller ekstraordinær, kan
ikke stille til samme verv på nytt før neste ordinære generalforsamling.

\subsubsection{}
Tilstedeværende stemmeberettigede kan komme med benkeforslag på kandidater til
stillinger som skal velges på generalforsamlingen fra og med
generalforsamlingen er satt, til avstemning holdes for den aktuelle stillingen.
Benkede kandidater kan velge å ikke stille til valg.

\subsection{Øvrig om generalforsamlingen}
\subsubsection{}
Hovedstyret skaffer en utenforstående revisor. Denne reviderer alle foreningens
regnskap for siste regnskapsår (spesifisert i § 10.1) før ordinær
generalforsamling på våren.  Kassererne legger fram resultatet til foreningens
godkjennelse på generalforsamlingen.

\subsubsection{}
På ordinær generalforsamling skal det sittende Hovedstyret legge fram en
redegjørelse for foreningens aktiviteter siden siste ordinære
generalforsamling.

\subsubsection{}
Det nye Hovedstyret presenteres på generalforsamlingen, og må godkjennes ved
kvalifisert flertall. Dersom det nye Hovedstyret underkjennes, må det eksisterende
Hovedstyret komme med nytt forslag til styre innen én -1- uke.

\subsubsection{}
Det nye Hovedstyret trer i kraft fra og med 15. april, men datoen kan flyttes opp til to
-2- uker hvis Hovedstyret ser dette som hensiktsmessig.

\subsubsection{}
Hovedstyret skal legge frem referat fra generalforsamlingen. Referatet skal
legges fram senest én -1- uke etter generalforsamlingen og skal publiseres på
foreningens hjemmeside.

\subsubsection{}
Bare generalforsamlingen kan opprette og nedlegge komiteer. Forslag om dette
skal være Hovedstyret i hende minst to -2- uker før annonsert
generalforsamling. Hovedstyret skal offentliggjøre dette senest én -1- uke før
generalforsamlingen finner sted. Forslag skal inneholde en begrunnelse for
opprettelsen/nedleggelsen. Dersom en komité skal opprettes skal forslaget også
inneholde en beskrivelse av komiteens tenkte virke.
